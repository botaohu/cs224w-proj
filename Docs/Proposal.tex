\documentclass[11pt,letterpaper]{article}

\newtheorem{theorem}{Theorem}
\newtheorem{corollary}{Corollary}
\newtheorem{lemma}{Lemma} 
\newtheorem{claim}{Claim}
\newtheorem{fact}{Fact}
\newtheorem{definition}{Definition}
\newtheorem{assumption}{Assumption}
\newtheorem{observation}{Observation}
\newtheorem{example}{Example}

\usepackage{epsfig}
\usepackage{graphicx}
\usepackage{amsmath}
\usepackage{amssymb}
\usepackage{enumerate}
\usepackage[margin=0.75in]{geometry}

\oddsidemargin 0in
\evensidemargin 0in
\textwidth 6.5in
\topmargin -0.5in
\textheight 9.0in
\abovecaptionskip 0in

\def\@maketitle
   {
   \newpage
   \null
   \vskip .375in
   \begin{center}
      {\Large \bf \@title \par}
      % additional two empty lines at the end of the title
      \vspace*{24pt}
      {
      \large
      \lineskip .5em
      \begin{tabular}[t]{c}
         \ifcvprfinal\@author\else Anonymous CVPR submission\\
         \vspace*{1pt}\\%This space will need to be here in the final copy, so don't squeeze it out for the review copy.
Paper ID \cvprPaperID \fi
      \end{tabular}
      \par
      }
      % additional small space at the end of the author name
      \vskip .5em
      % additional empty line at the end of the title block
      \vspace*{12pt}
   \end{center}
   }

\def\abstract
   {%
   \centerline{\large\bf Abstract}%
   \vspace*{12pt}%
   \it%
   }

\def\endabstract
   {
   % additional empty line at the end of the abstract
   \vspace*{12pt}
   }


\begin{document}
%%%%%%%%% TITLE
\title{CS224W Course Project Proposal: A survey of Network Alignment}

\author{Danqi Chen\\
Stanford University\\
{\tt\small danqi@stanford.edu}
% For a paper whose authors are all at the same institution,
% omit the following lines up until the closing ``}''.
% Additional authors and addresses can be added with ``\and'',
% just like the second author.
% To save space, use either the email address or home page, not both
\and
Botao Hu\\
Stanford University\\
{\tt\small botaohu@stanford.edu}
%
\and
Shuo Xie\\
Stanford University\\
{\tt\small shuoxie@stanford.edu}
}

\maketitle
\thispagestyle{empty}

\maketitle

%%%%%%%%% ABSTRACT
\begin{abstract}
   This document is a project proposal for the CS231A open course project. It details our plans for contributing to current research in real-time object tracking. Possible datasets, algorithms, readings  and evaluation methods are reviewed.
\end{abstract}

\section{Problem Statement}

Several attempts have recently been made to improve real-time object tracking in a sequence of frames by using a detector in addition to the tracker (Kalal  \cite{kalal2012tracking}, Pernici  \cite{pernicifacehugger}, Nebehay  \cite{nebehay2011evaluation}). The main goals of the detector are to prevent the tracker from drifting away from the object, and recover tracking after an occlusion. Since the only prior knowledge about the object is a bounding box in the initial frame, the detector must be trained online. In order to build such a system, two critical challenges must be addressed:
\begin{enumerate}
\item
finding an efficient feature-extraction algorithm to perform detection on thousands of subwindows in each frame
\item
using a powerful learning strategy to update the template used by the detector
\end{enumerate}
In addition, the solutions to these two problems are dependent on each other, and as such, they must be designed so as to fit into a single system.

The goal of this project is to investigate new algorithms for 1.\ and 2.\ and try to find improvements in terms of:
\begin{itemize}
\item
robustness of tracking (good performance with a wide range of objects, tolerance to poor video quality such as camera blur, low resolution, low frame rate, etc)
\item
efficiency (time and space complexity)
\end{itemize}
For demonstration purposes, some additional features could be introduced, such as simultaneous tracking of multiple objects.

\section{Algorithms}

Several feature extraction algorithms can be used for template matching. Feature descriptors such as FREAK (Vandergheynst \cite{vandergheynst2012freak}), BRISK (Leutenegger \cite{leutenegger2011brisk}) and ORB (Rublee  \cite{rublee2011orb}) could help speed up the template matching process.

We would like to investigate a deep learning approach for improving the learning step. Zou, W. \cite{zou} should be a good reference for designing our solution.


\section{Readings}

The papers mentioned in the references section of this document will provide context and background for this project.


\bibliographystyle{abbrv}
\bibliography{na}

\section{Appendix}

\end{document}
